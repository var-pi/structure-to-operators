\documentclass{article}
\usepackage{amsmath,amssymb,physics,mathtools}
\title{Hamiltonian mechanics}
\author{Stefan Ehin}

\begin{document}

\maketitle

A Hamiltonian is a smooth function
\[
    H: T^*Q \times \mathbb{R} \rightarrow \mathbb{R}.
\]
In standard mechanical systems
\[
    H(q,p,t) = T(q,p) + V(q,t)
\]
though in general $H$ need not be energy. In canonical coordinates $(q^i,p_i)$ Hamilton’s equations are
\[
    \dot{p}_i = -\pdv{H}{q^i} \quad \dot{q}^i = \pdv{H}{p_i}.
\]
Thus $H$ defines a vector field \(X_H \in \Gamma(TT^*Q)\) that generates a (local) one-parameter group of symplectomorphisms.

The cotangent bundle carries a canonical 1-form (Liouville / tautological form)
\[
    \theta \coloneqq p_i dq^i.
\]
Its exterior derivative gives the canonical symplectic form
\[
    \omega \coloneqq -d\theta = dq^i \wedge dp_i.
\]
The Hamiltonian vector field is defined implicitly by
\[
    \iota_{X_H}\omega = dH.
\]
Because $\omega$ is nondegenerate, this uniquely determines $X_H$.

Symplectic structure induces a Poisson bracket
\[
    \{F, G\} \coloneqq \omega(X_F,X_G)
\] that is handy for instance for calculating the total derivative of an observable
\[
    \frac{dF}{dt} = \{F,H\} + \frac{\partial F}{\partial t}.
\]
This implies that if \(\{F,H\} = 0\) and \(\pdv{F}{t}=0\) then an observable \(F\) is conserved along the flow.

\section*{Insights for Operator Learning}

The natural object is the flow generated by a symplectic vector field on the phase space. This converts dynamics into an operator semigroup acting on states.

By providing surrogate $H_\theta$ and defining $X_{H_\theta} \coloneqq J\nabla H_\theta$ we get a surrogate for $X_H$ whose continuous flow preserves $H_\theta$ and the symplectic form.

\end{document}
