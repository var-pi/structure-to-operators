\documentclass{article}
\usepackage{amsmath,amssymb,physics,mathtools}
\title{Hamiltonian mechanics}
\author{Stefan Ehin}

\begin{document}

\maketitle

In standard mechanics, Hamiltonian \(H: T^*Q \times \mathbb{R} \rightarrow \mathbb{R}\) coincides with total energy \(T+V\). In coordinates \((p, q) \in T^*Q\) it induces
\[
    \dot{\mathbf{p}} = -\pdv{H}{\mathbf{q}} \quad \dot{\mathbf{q}} = \pdv{H}{\mathbf{p}}.
\]
Hence given \(H\) we obtain an evolution vector field \(X_H \in \Gamma(TT^*Q)\). In a coordinate-free manner we can state the dynamics more concisely
\[
    \iota_{X_H}\omega = dH
\]
where \(\omega \in \Omega^2(T^*Q) \) is some symplectic form. Note that \(T^*Q\) has a tautological 1-form
\[
    \theta \coloneqq p_i dq^i \in \Omega(T^*Q)
\]
and hence also a canonical symplectic form
\[
    -d\theta = dq^i \wedge dp_i.
\]
Symplectic structure induces a Poisson bracket
\[
    \{F, G\} \coloneqq \omega(X_F,X_G)
\] that is handy for instance for calculating the total derivative of an observable
\[
    \frac{dF}{dt} = \{F,H\} + \frac{\partial F}{\partial t}.
\]
This implies that if \(\{F,H\} = 0\) and \(\pdv{F}{t}=0\) then an observable \(F\) is conserved along the flow.

\end{document}
