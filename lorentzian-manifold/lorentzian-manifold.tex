\documentclass{article}
\usepackage{amsmath,amssymb,mathtools}
\title{Lorentzian manifold}
\author{Stefan Ehin}

\begin{document}

\maketitle

The motion of a particle can be described by a smooth curve on a Lorentzian spacetime manifold $(M,g)$. Such curve
\[
    \gamma:I\rightarrow M
\]
is called a \textbf{worldline} and is physically admissible for a massive particle iff in $(-+++)$ signature convention  
\[
    \forall \tau \in I \ g(\dot{\gamma}(\tau), \dot{\gamma}(\tau)) < 0
\]
and $\dot\gamma$ lies in the chosen future light cone at each point.

A spacetime curve is more fundamental than a spatial trajectory parameterized by time, since a spacetime point has no distinguished time coordinate. To represent motion as x(t), one must choose a foliation of spacetime by spacelike hypersurfaces and thereby introduce a time function.

\textbf{Lorentz transformations} relate different inertial coordinate systems arising from different choices of spacetime splitting. In Minkowski spacetime, a Lorentz transformation is an isometry of the Lorentzian metric, and therefore preserves the light cone structure.

\section*{Insights for Operator Learning}

Consider a problem
\[
    \partial_t u = L u.
\]

If $L$ is a constant-coefficient, translation-invariant differential operator on $\mathbb{R}^d$, then the solution operator satisfies
\[
    S(t)u_0 = G_t \star u_0.
\]
For hyperbolic equations, causality implies finite propagation speed
\[
    \text{supp} \ G_t \subset B_{ct}(0).
\]
This is exemplified by d’Alembert’s formula for the 1D wave equation. It is also fascinating that from semigroup theory
\[
    S(t) = e^{tL}
\]
and hence equivarience with $L$ leads to equivarience with $S(t)$:
\[
    L(g \cdot u) = g \cdot (Lu) \Rightarrow S(t)(g \cdot u_0) = g \cdot (S(t)u_0)
\]
where $g \cdot u(x) \coloneqq u(g^{-1}x)$ denotes the pullback action of $g$ on functions.

Note also that the spacetime formulation replaces the evolution problem with a geometric problem of finding integral curves of a vector field on an appropriate configuration manifold.

\end{document}
